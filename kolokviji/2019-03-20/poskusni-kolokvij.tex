\documentclass[arhiv, 10pt]{{izpit}}
\usepackage{{fouriernc}}
\usepackage{{minted}}
\usepackage{{multicol}}
\usepackage{{paralist}}
\usepackage{{inconsolata}}
\usepackage[T1]{{fontenc}}
\usepackage{{xcolor}}
\usepackage[most]{{tcolorbox}}
\definecolor{{light-red}}{{rgb}}{{1,0.5,0.5}}
\usemintedstyle{{colorful}}
\setlength{{\columnsep}}{{30pt}}
\tcbset{{enhanced jigsaw,size=tight,colback=light-red,boxrule=0pt,extrude by=1pt,rounded corners,interior style={{opacity=0.3}}}}

\begin{{document}}

\izpit{{Uvod v programiranje: poskusni kolokvij}}{{20.\ marec 2019}}{{
  Pri vsaki nalogi obkrožite črko pred pravilnim odgovorom ali vpišite pravilno vrednost v ustrezen prostor. \\
  Čas reševanja je 30 minut. Veliko uspeha!
}}
\newcommand{{\inlinepy}}[1]{{\mintinline{{python}}{{#1}}}}
\newcommand{{\answerbox}}[1]{{\framebox{{\vphantom{{\large M}}\hspace{{#1cm}}}}}}
\newcommand{{\wrongbox}}[1]{{\tcbox{{\vphantom{{M}}#1}}}}
%%%%%%%%%%%%%%%%%%%%%%%%%%%%%%%%%%%%%%%%%%%%%%%%%%%%%%%%%%%%%%%%%%%%%%%

%%%%%%%%%%%%%%%%%%%%%%%%%%%%%%%%%%%%%%%%%%%%%%%%%%%%%%%%%%%%%%%%%%%%%%%
\naloga
\begin{{multicols}}{{2}}
\noindent
Program na desni je izjemno grdo napisan in vsebuje cel kup napak (v zadnji vrstici pri \inlinepy{{n}} manjka \inlinepy{{return}}). Natanko \textbf{{tri}} od osenčenih napak so \textbf{{sintaktične}}. Obkrožite jih.

\columnbreak

\begin{{minted}}[baselinestretch=1.2,escapeinside=||]{{python}}
def fakulteta|\wrongbox{{[n]}}|:
    if|\wrongbox{{    }}|n|\wrongbox{{==}}|0:
        return |\wrongbox{{(1)}}|
    |\wrongbox{{   }}|else|\wrongbox{{  }}|
        |\wrongbox{{n}}| * |\wrongbox{{fakotleta}}|(n - |\wrongbox{{2}}|)
\end{{minted}}
\end{{multicols}}

%%%%%%%%%%%%%%%%%%%%%%%%%%%%%%%%%%%%%%%%%%%%%%%%%%%%%%%%%%%%%%%%%%%%%%%
\naloga
Kaj izračuna spodnja funkcija?

\begin{{multicols}}{{2}}
  \begin{{minted}}{{python}}
  def f(sez):
      v = 0
      for x in sez:
          if x > 0:
              v += x
      return v > 0
  \end{{minted}}


  \begin{{enumerate}}[(a)]
    \item ali seznam vsebuje kakšno pozitivno število
    \item število pozitivnih števil v seznamu
    \item ali seznam vsebuje samo pozitivna števila
    \item vsoto vseh pozitivnih števil v seznamu
  \end{{enumerate}}
\end{{multicols}}

%%%%%%%%%%%%%%%%%%%%%%%%%%%%%%%%%%%%%%%%%%%%%%%%%%%%%%%%%%%%%%%%%%%%%%%
\naloga
Katera izmed spodnjih funkcij izračuna produkt vseh elementov danega seznama?

\begin{{multicols}}{{2}}
\begin{{enumerate}}[(a)]
\item \begin{{minted}}{{python}}
def prod(sez):
    if sez == []:
        return 1
    else:
        return sez[0] * prod(sez[1:])
\end{{minted}}

\item \begin{{minted}}{{python}}
def prod(sez):
    if sez == []:
        return 1
    else:
        return prod(sez[0] * sez[1:])
\end{{minted}}

\item \begin{{minted}}{{python}}
def prod(sez):
    if sez[0] == 1:
        return prod(sez[1:])
    else:
        return prod(sez)
\end{{minted}}

\item \begin{{minted}}{{python}}
def prod(sez):
    if sez[0] == 0:
        return 0
    else:
        return sez[0] * prod(sez[1:])
\end{{minted}}
\end{{enumerate}}
\end{{multicols}}

%%%%%%%%%%%%%%%%%%%%%%%%%%%%%%%%%%%%%%%%%%%%%%%%%%%%%%%%%%%%%%%%%%%%%%%
\naloga
Katere vrstice izpiše klic \inlinepy{{print(f(g(3)))}}, če sta funkciji \inlinepy{{f}} in \inlinepy{{g}} definirani kot spodaj?

\begin{{multicols}}{{2}}
\begin{{minted}}{{python}}
def f(x):
    print(x)
    return x + 10
\end{{minted}}

\begin{{minted}}{{python}}
def g(x):
    return 2 * x
    print(x)
\end{{minted}}

\begin{{enumerate}}[(a)]
  \item \inlinepy{{3}}, \inlinepy{{6}}, \inlinepy{{16}}\hfill
  \item \inlinepy{{6}}, \inlinepy{{16}}\hfill
  \item \inlinepy{{16}}, \inlinepy{{6}}, \inlinepy{{3}}\hfill
  \item \inlinepy{{16}}, \inlinepy{{6}}
\end{{enumerate}}
\end{{multicols}}



%%%%%%%%%%%%%%%%%%%%%%%%%%%%%%%%%%%%%%%%%%%%%%%%%%%%%%%%%%%%%%%%%%%%%%%
\naloga
Dopolnite spodnji program tako, da bo na koncu v spremenljivki \inlinepy{{m}} shranjeno največje od števil \inlinepy{{x}}, \inlinepy{{y}} in \inlinepy{{z}}:

\begin{{minted}}[baselinestretch=1.2,escapeinside=||]{{python}}
m = |\answerbox{{1.5}}|
if x < y:
    if |\answerbox{{1.5}}| < |\answerbox{{1.5}}|:
        m = z
    else:
        |\answerbox{{3}}|
\end{{minted}}

%%%%%%%%%%%%%%%%%%%%%%%%%%%%%%%%%%%%%%%%%%%%%%%%%%%%%%%%%%%%%%%%%%%%%%%
\naloga
Katera izmed spodnjih funkcij vrača drugačen rezultat kot ostale?

\begin{{multicols}}{{2}}
\begin{{enumerate}}[(a)]
\item \begin{{minted}}{{python}}
def f(x, y):
    if x >= y:
        return x * y
    return x + y
\end{{minted}}

\item \begin{{minted}}{{python}}
def f(x, y):
    if x < y:
        return x + y
    elif x >= y:
        return x * y
\end{{minted}}

\item \begin{{minted}}{{python}}
def f(x, y):
    return x + y
    if x >= y:
        return x * y
\end{{minted}}


\item \begin{{minted}}{{python}}
def f(x, y):
    if x >= y:
        return x * y
    if x < y:
        return x + y
\end{{minted}}
\end{{enumerate}}
\end{{multicols}}

%%%%%%%%%%%%%%%%%%%%%%%%%%%%%%%%%%%%%%%%%%%%%%%%%%%%%%%%%%%%%%%%%%%%%%%
\naloga
Pri vsakem od programov na desni zapišite ustrezno črko programa na levi, ki ima pri začetni vrednosti \inlinepy{{i = 0}} enak izpis.

\begin{{multicols}}{{2}}
\begin{{enumerate}}[(a)]
\item \begin{{minted}}{{python}}
while i > 10:
    i = 10
    print(i)
\end{{minted}}

\item \begin{{minted}}{{python}}
if i < 10:
    print(i)
    i = 10
\end{{minted}}
\item \begin{{minted}}{{python}}
while i < 10:
    i += 1
print(i)
\end{{minted}}
\end{{enumerate}}
\begin{{itemize}}[\answerbox{{0.5}}]
\item \begin{{minted}}{{python}}
if i > 10:
    print(i)
    i = 10
\end{{minted}}
\item \begin{{minted}}{{python}}
print(i)
while i < 10:
    i += 1
\end{{minted}}
\item \begin{{minted}}{{python}}
while i < 10:
    i = 10
    print(i)
\end{{minted}}
\end{{itemize}}
\end{{multicols}}
%%%%%%%%%%%%%%%%%%%%%%%%%%%%%%%%%%%%%%%%%%%%%%%%%%%%%%%%%%%%%%%%%%%%%%%
\naloga

\begin{{multicols}}{{2}}
\noindent
Definirana naj bo funkcija \inlinepy{{f}}:

\begin{{minted}}{{python}}
def f(x, y):
    z = 0
    while x > 0:
        if x % 10 > y:
            z += 1
        x = x // 10
    return z
\end{{minted}}
\columnbreak
Napišite katerikoli števili \inlinepy{{x}} in \inlinepy{{y}}, za kateri klic \inlinepy{{f(x, y)}} vrne rezultat \inlinepy{{5}}.

\begin{{minted}}[baselinestretch=1.2, escapeinside=||]{{python}}
  x = |\answerbox{{3}}|
  y = |\answerbox{{3}}|
\end{{minted}}
\end{{multicols}}


%%%%%%%%%%%%%%%%%%%%%%%%%%%%%%%%%%%%%%%%%%%%%%%%%%%%%%%%%%%%%%%%%%%%%%%
\naloga

\begin{{multicols}}{{2}}
\noindent
S številkami od $0$ do $4$ označite vrstni red, v katerem moramo izvesti ukaze na desni, da bo na koncu v spremenljivki \inlinepy{{z}} shranjen niz \inlinepy{{'mama'}}?
\columnbreak
\begin{{minted}}[baselinestretch=1.2, escapeinside=||]{{python}}
|\answerbox{{0.5}}| z += w
|\answerbox{{0.5}}| z += 'm'
|\answerbox{{0.5}}| z = ''
|\answerbox{{0.5}}| z += 'a'
|\answerbox{{0.5}}| w = z
\end{{minted}}
\end{{multicols}}

\end{{document}}
